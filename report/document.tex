\documentclass[11pt]{article}
\usepackage[top=2.54cm, bottom=2.54cm, left=2.75cm, right=2.75cm]{geometry}
\linespread{1.2}
\setlength{\parindent}{0cm}
\usepackage{amsmath}
\usepackage{amssymb}
\usepackage{amsthm}
\usepackage{datetime}

\begin{document}
	\pagenumbering{arabic}
	
	\begin{titlepage}
		\begin{center}
			{\Huge Monte Carlo Simulation of Phase Transitions}\\[0.5cm]
			\textit{Haodong Chang and Tavis O'Reilly}\\[0.3cm]
			\textit{11505303 and 10903943}\\[0.3cm]
			Department of Physics and Astronomy\\[0.3cm]
			University of Manchester\\[0.3cm]
			PHYS20872 Theory Computing Project\\[0.3cm]
			\shortmonthname[\the\month]  \the\year \\[4cm]
			
		\end{center}
		
		{\Large \textbf{Abstract}}\\[0.3cm]
		
		
	\end{titlepage}
	
	\pagenumbering{gobble}
	\clearpage
	\pagenumbering{arabic}
	\setcounter{page}{2}
	
	\newpage
	
	\section{Introduction}
	
	Brief introduction to the problem.

	Our discussion over phase transitions will be based on the Ising model, 
	which is a one of the few models that can be solved analytically in two dimensions.

	(To be added: definition of phase transitions, the importance of the Ising model, etc.)
	
	\section{Theory}

	Let's consider a simple model of spins on a lattice.
	Each spin can be in one of two states, up or down, which we will represent by $s_i = \pm 1$.

	Every spin interacts with its nearest neighbours, and each pair contributes $-J s_i s_j$ to the energy, 
	where $J$ is a constant representing the strength of the interaction.

	There is also an external magnetic field, represented by $h$.
	Every spin in the magnetic field contributes $-h s_i$ to the total energy.

	A figure showing 1D and 2D lattices with spins (to be added).

	Thus the total energy of the system is given by
	\begin{equation}
		H(\{s_i\}) = -J \sum_{\langle i,j \rangle} s_i s_j - h \sum_i s_i
	\end{equation}
	where the first sum is over all pairs of nearest neighbours, and the second sum is over all spins.
	We may call the total energy as Hamiltonian in the following.
	
	Our calculation and simulation will be dealt in the canonical ensemble, where the temperature $T$ is fixed.

	By now, we have encountered every ingredient of the Ising model: 
	shape of the lattice $\langle i,j \rangle$, interaction strength $J$, external magnetic field $h$, and temperature $T$.
	
	Given these, we can calculate the partition function of the system, which is the key to all thermodynamic properties.
	\begin{equation}
		Z = \sum_{\{s_i\}} e^{-\beta H(\{s_i\})}, \quad \beta = \frac{1}{k_B T}
	\end{equation}
	The canonical ensemble assigns a probability to each state $\{s_i\}$ given by
	\begin{equation}
		P(\{s_i\}) = \frac{e^{-\beta H(\{s_i\})}}{Z}
	\end{equation}
	As you can see, states with lower energy are more likely to be found.

	Then we can discuss the meaning of $J$ and $h$.
	\begin{itemize}
		\item If $J > 0$, neighbours with parallel spins have lower energy, so the system tends to be aligned, which is called ferromagnetic.
		\item If $J < 0$, neighbouring spins tend to be anti-parallel, which is called anti-ferromagnetic.
	\end{itemize}
	\begin{itemize}
		\item If $h > 0$, spins prefer to be $s_i = +1$.
		\item If $h < 0$, spins prefer to be $s_i = -1$.
	\end{itemize}
	We can see that spins tend to align with the magnetic field, which is called paramagnetic.

	(Briefly mention the concept of phase transitions. Add or not?)

	In the following of the theory section, we will start from some simple cases, and then generalize to more complex ones.

	\subsection{Ising Model without interaction ($J=0$)}

	When there is no interaction between spins, the shape of the lattice doesn't matter at all, and the Hamiltonian becomes
	\begin{equation}
		H(\{s_i\}) = -h \sum_i s_i
	\end{equation}
	which is simply the energy of a collection of spins in a magnetic field.

	Then the partition function is
	\begin{equation}
		\begin{aligned}
			Z &= \sum_{\{s_i\}} e^{-\beta H(\{s_i\})} = \sum_{\{s_i\}} e^{\beta h \sum_i s_i} = \prod_i \sum_{s_i=\pm1} e^{\beta h s_i} \\
			&= \prod_i (e^{\beta h} + e^{-\beta h}) = (e^{\beta h} + e^{-\beta h})^N = (2\cosh(\beta h))^N
		\end{aligned}
	\end{equation}

	The energy of the system can be calculated as
	\begin{equation}
		E \equiv \langle H \rangle = -\frac{\partial}{\partial \beta} \ln Z = - N h \tanh(\beta h)
	\end{equation}

	Here because of the absence of interaction, the relation between energy and magnetisation is simple, and we have
	\begin{equation}
		M \equiv \langle \sum_i s_i \rangle = -\frac{E}{h} = N \tanh(\beta h)
	\end{equation}

	\subsection{1D Ising Model without external field ($h=0$)}

	In this case, the Hamiltonian becomes
	\begin{equation}
		H(\{s_i\}) = -J \sum_{i} s_i s_{i+1}
	\end{equation}
	
	\section{Implementation}
	
	Brief description of how the model was built into the code and how data was obtained from this.
	
	\section{Results}
	
	The results obtained including some good plots. Compare to the theory.
	
	\section{Conclusion}
	
	Conclude results.
	
\end{document}